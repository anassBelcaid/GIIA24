\documentclass[palatino,code]{ensaexam}
\usepackage{minted}
\begin{document}
\ModuleName{Linux et Logiciel libre}
\ExamCode{GIIA423}
\ExamPeriod{Spring 2022}
\TimeAllowed{90}
\Logo{
\begin{center}
  \includegraphics[width=3cm, height=3cm]{ENSA-SAFI.png}
\end{center}
}
\Instructions{
  \begin{itemize}
    \item Vous avez {\bf\TheTimeAllowed\; minutes}. 
    \item Verifier que vous disposez de toutes les pages. 
    \item L'échange d'outils est strictement \textbf{interdit}.
  \end{itemize}
  
}
\MakeHeading
\vspace*{1cm}

 \begin{questions}

   % Question script with for {{{ %
   \titledquestion{Scripting I (**)}
   Dans ce problème, on se propose d'écrire un \textbf{script} qui vérifie
   l'état d'un dossier et affiche certains statistiques sur ces fichiers
   executables.\\

   \begin{itemize}
     \item Pour les trois questions suivantes, vous pouvez écrire un
       \textbf{seul} script qui regroupe les trois réponses.\\[4pt]
   \end{itemize}

   \begin{parts}
     \part[2] Écrire un le script \textbf{script.sh} qui vérifie si un dossier
     donné comme premier \textbf{argument} existe. Si non le script doit afficher un
     message d'erreur. 
     \part[1] Modifier votre script, pour qu'il affiche le nombre de fichiers
     \textbf{exécutables} qui figure dans ce dossier. 
     \part[2] Le script maintenant doit afficher tous ces fichiers
     \textbf{executables} triés selon le nombre du mot \textbf{echo} qu'il
     contiennent.
   \end{parts}
   \vspace*{.5cm}
   % }}} Question script with for %
% Data Wrangling {{{ %
\titledquestion{Manipulation Donnes(**)}
Pour cet exercice, vous devez utiliser votre maitrise de la commande
\textbf{grep} pour afficher des statistiques de votre dictionnaire qui figure
dans \textbf{/usr/share/dict/words}

\begin{parts}
  \part[2] Donner une commande, qui calcule le nombre de mots qui contient au
  moins trois lettres \textbf{a} ( qui ne sont pas forcement successives).
  \part[3] Pour ces mots, on cherche a identifier les \textbf{deux Derniers
  caractères} de chaque mots.
  \begin{itemize}
    \item Ecrire alors une fonction qui affiche les trois deux lettres les plus
      répondues. Par exemple votre fonction doit afficher:

      \begin{minted}{bash}
       101 an
       63 ns
       54 as
      \end{minted}

      Identifiant que $\mathbf{101}$ mots se termine par \textbf{an}, et
      $\mathbf{63}$ se terminent par \textbf{ns}.
  \end{itemize}
  \part[3] Comme vous avez appris, un fichier \textbf{HTML}, est basé sur la
  syntaxe des \textbf{tag}.  Comme exemple:

      \begin{minted}{html}
<article>
<title>About the Web</title>

<para>
This is an article about the World Wide Web.
The World Wide Web is a collection of documents that are linked to
one another. The Web is <emphasis>not</emphasis> the same as the
Internet. The Internet is a world-wide network of networks, and it
does far more than simply serve up Web pages.
</para>
\end{minted}
\vspace*{.5cm}
\begin{itemize}
  \item Écrire une commande, qui supprime tous ces les \textbf{tags} d'un
    fichier donné.
\end{itemize}

\end{parts}
\vspace*{1cm}
% }}} Data Wrangling %
% Fold description {{{ %
  \titledquestion{Luhn (***)}
  Dans cet exercice, on doit écrire un script qui détermine si un nombre est
  valide ou non selon la formule de \textbf{Luhn}. C'est un algorithme de
  hachage simple qui est utilise dans une variétés de cartes de crédits
  sociales.

  \begin{parts}
    \part[7] Vous devez écrire un script qui vérifie si le nombre en argument est
    valide ou non.
  \end{parts}

  Pour valider un nombre vous devez:

  \begin{itemize}
    \item Détruire tous les espaces du code, pour obtenir que des chiffres. Par
      exemple on suppose qu'on as:
      \begin{minted}{bash}
       4539 3195 0343 6467 
      \end{minted}
    \item Maintenant vous devez doubler chaque chiffre en commencent par la
      droite. Par exemple pour note nombre,  les chiffres qui \textbf{ne sont
      pas marqués} par \textbf{\_} seront doublés.  

      \begin{minted}{bash}
       4_3_ 3_9_ 0_4_ 6_6_
      \end{minted}
      Si en doublant la valeur dépasse 9, on retranche alors 9 du résultat.
      Ainsi on trouve la valeur:
      \begin{minted}{bash}
       8569 6195 0383 3437
      \end{minted}
      Vous remarquez que le premier $4$ est remplacé par $4\times 2 = 8$, Pour
      le $6$, on obtient $6\times2 = 12$ et puisque c'est supérieur a $9$ on
      retranche 9. On obtient alors: $12 - 9 = 3$.
    \item Pour l'étape finale on calcule la somme de chiffres on verifie si
      cette somme est divisible par $10$.
      \begin{minted}{bash}
8+5+6+9+6+1+9+5+0+3+8+3+3+4+3+7 = 80
      \end{minted}
      Ainsi ce nombre est \textbf{valide}.
  \end{itemize}
  
  Pour illustrer ce mécanisme, on vous montre un deuxième nombre:
  \begin{enumerate}
    \item valeur initiale $8273\;1232\;7352\;0569$.
    \item Doubler les chiffres $7253\;2262\;5312\;0539$
    \item Calculer la somme $7+2+5+3+2+2+6+2+5+3+1+2+0+5+3+9 = 57$
    \item Cette somme n'est pas divisible par $10$, alors le nombre n'est pas
      valide.
  \end{enumerate}
% }}} Fold description %

  
 \end{questions}
 
 
\end{document}

